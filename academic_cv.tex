% LaTeX resume using res.cls
\documentclass[margin, line]{res}

\usepackage{marvosym}
\usepackage{wasysym}
\usepackage{hyperref}
\usepackage{enumitem}
\usepackage{multicol}
\usepackage{etaremune}

\sectionskip=1.5ex plus 1ex minus -.2ex % values stolen from LaTeX
\begin{document}

\name{Jeremiah Cha}
\begin{resume}

\section{Research\\Interests}
American Politics, Congress, Representation, Legislative Organization\\
Political Methodology

\section{Contact}
Harvard Department of Government \hfill \Letter: \href{mailto:jeremiahcha@g.harvard.edu}{jeremiahcha@g.harvard.edu}\\
1737 Cambridge Street \hfill \Mundus: \href{httsp://www.jeremiahcha.com}{www.jeremiahcha.com}\\
Cambridge, MA 02138 \hfill \Mundus: \href{https://orcid.org/0000-0003-3199-3386}{orcid.org/0000-0003-3199-3386}

\section{Education}
Ph.D. in Government, Harvard University \hfill Expected 2026\\
\hspace*{5mm}Graduate Affiliate, Center for American Political Studies

B.A. in Political Science, UC San Diego \hfill 2019\\
\hspace*{5mm}\textit{magna cum laude}, department honors

\section{Grants and \\Awards}
V.O. Key Fellowship \hfill 2020 - \\
Phi Beta Kappa \hfill 2019\\
High Honors Distinction, UCSD Political Science \hfill 2019\\
Pi Sigma Alpha \hfill 2017


\section{Selected Works \\in Progress}
\begin{enumerate}
	\item ``A Model of Congressional Review''
	\item ``The Electoral Connection and Congressional Committees''
	\item ``The Direct Primary Election'' (with James M. Snyder, Jr.)
	\item ``The President and the Cardinals: Interbranch Bargaining and Subcommittee Influence in Federal Appropriations'' (with Jon Rogowski)
\end{enumerate}

\section{Data}
\begin{enumerate}
	\item Cha, Jeremiah, Shiro Kuriwaki, and James M. Snyder, Jr. (2021). ``Candidates in American General Elections.'' Harvard Dataverse. \textsc{doi}: \href{https://doi.org/10.7910/DVN/DGDRDT}{\texttt{10.7910/DVN/DGDRDT}}.
\end{enumerate}

\section{Research Experience}
Professor Stephen Ansolabehere, Harvard \hfill 2021 - \\
Pew Research Center, Global Attitudes \hfill 2019 - 2020\\
Professor Tom K. Wong, UC San Diego \hfill 2017 - 2019\\
Brookings Institution, Governance Studies \hfill Summer 2018

\section{Non-refereed Publications}

\begin{etaremune}
	\item Devlin, Kat, Regina Widjaya and Jeremiah Cha. 2020. ``For Global Legislators on Twitter, an Engaged Minority Creates Outsize Share of Content.'' Pew Research Center.
	\item Cha, Jeremiah. 2020. ``Fast facts about South Koreans’ views of democracy as legislative election nears.'' Pew Research Center.
	\item Cha, Jeremiah. 2020. ``People in Asia-Pacific regard the U.S. more favorably than China, but Trump gets negative marks.'' Pew Research Center.
	\item Huang, Christine and Jeremiah Cha. 2020. ``Russia and Putin receive low ratings globally.'' Pew Research Center.
	\item Fetterolf, Janell and Jeremiah Cha. 2020. ``Few in other countries approve of Trump’s major foreign policies, but Israelis are an exception.'' Pew Research Center.
	\item Wong, Tom K., Jeremiah Cha and Erika Villareal-Garcia. 2019. ``The Impact of Changes to the Public Charge Rule on Undocumented Immigrants Living in the United States.'' U.S. Immigration Policy Center (USIPC) at UC San Diego.
	\item Silver, Laura, Emily A. Vogels, Mara Mordecai, Jeremiah Cha, Raea Rasmussen and Lee Rainie. 2019. ``Mobile Divides in Emerging Economies.'' Pew Research Center.
	\item Cha, Jeremiah. 2017. ``Dreaming of DACA.'' Claremont Journal of Law and Public Policy.
\end{etaremune}


\section{Service}
Rules and Election Officer, Harvard Government GSA \hfill 2021 -\\
Diversity Committee Member, Harvard University \hfill 2020 -

\section{Invited Talks}
Pew Research Graduate School Panel \hfill June 2020

\section{Non-Academic Experience}
House of Representatives, Intern \hfill Summer 2017\\
House of Representatives, Intern \hfill Summer 2016\\
Chaminade College Prep, Debate Coach \hfill 2015 - 2018

\section{Skills}
Technical: R, Stata, SPSS, \LaTeX, HTML\\
Language: Korean (Limited Working)

\small{References available upon request} \hfill \small{\textit{Updated Sept 2021}}

\end{resume}
\end{document}
